\documentclass[12pt]{article}

\usepackage{graphicx}
\usepackage{paralist}
\usepackage{amsfonts}
\usepackage{listings}
\usepackage{url}

\oddsidemargin 0mm
\evensidemargin 0mm
\textwidth 160mm
\textheight 200mm
\renewcommand\baselinestretch{1.0}

\pagestyle {plain}
\pagenumbering{arabic}

\newcounter{stepnum}

\newcommand\CC{C\texttt{++}}

\title{Assignment 3}
\author{COMP SCI 2ME3 and SFWR ENG 2AA4}

\begin {document}

\maketitle

\begin{description}
\item [Assigned:] February 25, 2019
\item [Part 1 (Spec):]  March 1, 2019
\item [Receive Full Spec:] March 6, 2019
\item [Part 2 (Implement and Test):] March 22, 2019
\item [Last Revised:] \today
\end{description}

\noindent All submissions are made through git, using your own repo located at:\\

\texttt{https://gitlab.cas.mcmaster.ca/se2aa4\_cs2me3\_assignments\_2018/[macid].git}\\

\noindent where \texttt{[macid]} should be replaced with your actual macid.  The
time for all deadlines is 11:59 pm.  

\section{Introduction}

The purpose of this software design exercise is to design and implement a
portion of the specification for the game of Forty Thieves
(\url{https://greenfelt.net/fortythieves}).  For Part 1, you are given a partial
specification and asked to fill in the specification of the missing semantics.
Once the specification is complete, you will implement it for Part 2.  To have a
common interface across the class, to facilitate unit testing by the TAs, you
will implement the instructor provided specification, rather than your own.

The game of Forty Thieves uses two standard 52-card decks, for a total of 104
cards.  The set-up of the board and the rules are as follows:
\begin{itemize}
\item For the initial board, a tableau of 10 stacks of 4 cards are created, with
  the remaining cards still in the deck.  All of the tableau card stacks are
  face up and visible.
\item Above the tableau are 8, initially empty, foundation stacks.  There are
  two foundation stacks for each suit.
\item Cards can only be moved from the top of each tableau stack.  You may place
  any card in an empty tableau space.
\item The tableau cards can be moved from the top of one tableau to the top of
  another, by building down by suit, from King to Ace.
\item The foundation slots are built up by suit in the opposite order, from Ace
  to King.
\item Cards may be dealt one at a time from the remaining deck to the waste
  stack.
\item You may use the top card from the waste stack.  You may only go through
  the deck once.
\item The object of the game is to move all the cards to the foundation stacks.
\end{itemize}

You will complete the specifications for the modules described in the
specification file.  Your specifications should not involve writing algorithms
or pseudo-code.  The specifications should use discrete mathematics to specify
the desired properties.  That is, you should be writing a {\it descriptive}
specification as opposed to an {\it operational} specification.  Specifications
within a module are free to use access programs defined within the current
module or from another module that is used by the current module.  You should
use the provided local functions.  You do not have to add more local
functions, but you can, if you find them helpful.

All of your code should be written in \CC{}.  All code files should be documented
using doxygen.  Your specification/report (for Part 1) should be written using
\LaTeX.  Your code should follow the given specification exactly.  In
particular, you should not add public methods or procedures that are not
specified and you should not change the number or order of parameters for
methods or procedures.  If you need private methods or procedures, you can add
them by explicitly declaring them as private.

\section*{Part 1}

\section *{Step \refstepcounter{stepnum} \thestepnum}

Complete the specification by addressing the comments given in the specification
file \texttt{spec.tex}.  You should note that this specification uses a
``functional'' approach.  Rather than modify the state variables, the
specification calls for the creation of a new object with the new state.

\section *{Step \refstepcounter{stepnum} \thestepnum}

Write a critique of the interface for the modules in this project.  Is there
anything missing?  Is there anything you would consider changing?  Why?  Your
critique should appear as the last section of \texttt{spec.tex}.

\section *{Step \refstepcounter{stepnum} \thestepnum}

Push your spec, which will include the critique, in (\texttt{spec.tex} and
\texttt{spec.pdf}) to your GitLab project repo.  The report, including the
specifications, should be written in \LaTeX.  This step should be completed by
the deadline for Part 1 of the assignment.

\section*{Part 2}

\section *{Step \refstepcounter{stepnum} \thestepnum}

After the report has been submitted, you will be provided with a complete
specification for all of the modules.  Implement the modules in \CC{}.  Blank
versions of the required source files and header files will be pushed to your
personal repo.

\section *{Step \refstepcounter{stepnum} \thestepnum}

Experiment with the implementation.  Test the supplied \texttt{Makefile} rule
for {\tt experiment}.  The purpose of this rule is to provide a means for
``playing'' with the code as you develop it.

\section *{Step \refstepcounter{stepnum} \thestepnum}

Test the supplied \texttt{Makefile} rule for {\tt doc}.  This rule should
compile your documentation into an html and \LaTeX{} version.  Along with the
supplied \texttt{Makefile}, a doxygen configuration file is also given in your
initial repo.  You should not change these files.

\section *{Step \refstepcounter{stepnum} \thestepnum}

In the \texttt{test} folder you should have a corresponding test file for the
CardStack and GameBoard modules.  (You can determine the names
for the testing files.)  The testing framework being used is called catch, as
discussed in the tutorials.  The header file you need, along with the
\texttt{testmain.cpp} file, will already be pushed to your repo.  Each procedure
should have at least one test case.  For this assignment you are not required to
submit a lab report, but you should still carefully think about your rationale
for test case selection.  Please make an effort to test normal cases, boundary
cases, and exception cases.

The supplied makefile (named {\tt Makefile}) will have a rule named {\tt
  test}.  This rule should run all of your test cases.

\section *{Step \refstepcounter{stepnum} \thestepnum}

Push all of your code files to your GitLab project repo.  This step should be
completed by the deadline for Part 2 of the assignment.

\subsubsection*{Notes}

\begin{enumerate}
\item Please put your name and macid at the top of each of your source files.
\item Your program must work in the ITB labs on mills when compiled with its
  versions of g++ (Version 7), \LaTeX, doxygen and make.
\item So that you will have the correct version of g++, please add the
  following:

  \texttt{. /opt/rh/devtoolset-7/enable} 

  to the bottom of your \texttt{.bashrc} file on mills.  (An earlier version of
  g++ has to concurrently exist on mills; this allows you to switch to the
  version we are using.)
\item Many choices are available for containers to store sequences.  So
  that unit testing will work between submissions, we need to make a standard
  decision.  Therefore, please use a vector for your \CC{} implementation for the
  constructor for Stack:
\begin{lstlisting}
template <class T>
Stack<T>::Stack(std::vector<T> s)
{
  //details
}

\end{lstlisting}
\item Pointers should \emph{not} be used in your interfaces.  All methods should
  use pass by value and all methods should return a value.
\item Implement templates following the approach used in the tutorial.  That is,
  the CardT instance should be explicitly created in the \texttt{Stack.cpp}
  file.  The header file \texttt{CardStackT.h} will only need to be the
  appropriate typedefs.
\item Enumerated types, like SuitT, should be implemented using \texttt{enum} in
  \CC{}.
\item The specification shows RankT restricted to values between 1 and 13.  For
  your implementation you are not expected to enforce this.  You shoud use
  \texttt{unsigned short int}.
\item Natural numbers ($\mathbb{N}$) can be represented in \CC{} by
  \texttt{unsigned int}.
\item Tuple types, like CardT, should be implemented using \texttt{struct} in \CC{}.
\item For exception, you should use the standard exceptions for \CC{}
  (\url{http://www.cplusplus.com/reference/stdexcept/}).  The names of
  exceptions were chosen for the specification to make this straightforward.
  For the \CC{} implementation each exception name should be prefixed by
  \texttt{std::}.  For instance, popping from an empty stack should raise a
  \texttt{std::out\_of\_range} exception.
\item \textbf{Your grade will be based to a significant extent on the ability of
    your code to compile and its correctness.  If your code does not compile,
    then your grade will be significantly reduced.}
\item \textbf{Any changes to the assignment specification will be announced in
    class.  It is your responsibility to be aware of these changes.  Please
    monitor all pushes to the course git repo.}
\end{enumerate}

\end {document}
